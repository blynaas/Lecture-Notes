\documentclass[12pt,a4paper]{report}
\usepackage{graphicx}
\begin{document}

	\centering
	{\scshape\LARGE CLAS100 \par}
	{\scshape\Large Lecture 4\par}
	%{\Large\itshape Yaro Gorban\par}
	{\large \today\par}
	\vspace{1.5cm}
	There was a constitutional enmity(compare capitalism and communism)
	\begin{itemize}
	\item The democratic ideology (Athens)
	\item The communistic ideology (Sparta)
	\end{itemize}
	The Peloponnesian War, 431-404 BCE
	\begin{itemize}
	\item Fear of Athenian imperial(naval power)
	\item Sparta won the war
	
	\item The Classical period contains the high point the high and the low points of Greek civilization.
	\begin{itemize}
	\item The war vs. Persia was the high point
	\item The civil war between Sparta and Athens was the low point
	\end{itemize}
	\end{itemize}
	\textbf{After the Peloponnesian War}
	\begin{itemize}
	\item Hostilities continue, with the Corinthian War 395 BC
	\item There were shifting alliances
	\item Caused the rise of Macedon(ia)
	\end{itemize}
	\textbf{Macedon(ia)}
	\begin{itemize}
	\item Semi-Greek kingdom
	\item Had a monarchy
	\begin{itemize}
	\item powerful local elites
	\item Instability
	\item Interference of south Greece
	\item A lot of coups and assassinations
	\item Until Philip II of Macedon 359-336 BCE
	\end{itemize}
	\end{itemize}
	\textbf{Philip II}
	\begin{itemize}
	\item If not for him, Alexander the Great may have never had the empire he historically had
	\item Expanded Macedon's borders
	\item Military reforms
	\item Diplomat and marriage alliance
	\item Had a political marriage with Olympias of Epiros
	\item He was able to manipulate the Greeks to turn the states against one another
	\item Had a special relationship with Athens
	\item Once the his army was made Athens was scared
	\item Philip would have a way to make people feel safe
	\item Would send messages to Athens telling them he is not planning to attack
	\item By challenging Athens' ability to gather wheat there was finally a conflict
	\item The Battle of Chaironeia 338 BC
	\item Feigned Defeat to trap the Greeks
	\item The decided to move into Persia
	\item Was Assassinated in 336 BC
	\item Alexander The Great takes over
	\end{itemize}
	\textbf{Alexander The Great}
	\begin{itemize}
	\item The Battle of Isos defeats Darius III
	\item Named a town called Alexandria in Egypt
	\item Created self titled towns in many places
	\item He instituted Local Greek speaking Government
	\item Moved to Babylon and begins to take after the eastern Kings
	\item Crazed with power
	\begin{itemize}
	\item Greece felt reduced to a small state in a large empire
	\item Felt strange to need to drop to their knees, as they only did this for their Gods
	\end{itemize}
	\item Died or was Killed in Babylon 10 June 323 BC
	\item Historians are unsure if he died from a wound that got infected during battle, drank to much(very unpopular hypothesis) or was assassinated by a disgruntled General
	\item The people were not prepared for his death and had no succession plan(which was troubling)
	\end{itemize}
Next Lecture continues with the period following the death of Alexander the Great.
% Bottom of the page
	

\end{document}