\documentclass[12pt,a4paper]{report}
\usepackage{graphicx}
\usepackage{amsmath}
\usepackage{amsfonts}
\usepackage{amssymb, mathpazo}

\begin{document}

	\centering
	{\scshape\LARGE CLAS100 \par}
	{\scshape\Large Greek History Review \par}
	%{\Large\itshape Yaro Gorban\par}
	\vspace{1.5cm}

\textbf{Important Dates and Events and Places and Words}
\begin{itemize}
\item The Trojan War and "Agamemnon" around 1200 BCE
\item The First Mycenaeans in around 1600-1550 BCE
\item Knossos The city formed like a labyrinth(people had access to the Palace)
\item Height of Minoan Civ 2000-1450 BCE
\item Destruction of Minoan palaces 1450 BCE
\item Minoans and Mycenaeans had a lot of differences including language (Linear A for Minoans which was never Deciphered and Linear B which was Deciphered by Michael Ventris)
\item Minoans and Mycenaeans ethnically distinct
\item Myceneans Conquered Minoans 1450 BC
\item Mycenaean Civ ends 1200-1100 BCE 
\item Polis: City-State(a localized community, acropolis(High city) agora(marketplace)) Such as Athens
\item Polis is also: Citizen-State(citizen governance and citizen soldiers)
\item Aristo-kratia
\item Classical Athens: demo-kratia
\item Unity and Hellenism: Hellenic = Greek, Panhellenic = all Greek
\item Panhellenic celebrations and sacred sites like Delphi and Olympia
\item Persian Invasions: 490 BCE: the expedition of Darius(Marathon), 480 BCE: the expedition of Xerxes(Thermopylai, Salamis), this was the height of Greek cooperation.
\item Athens was a democracy
\item Sparta was oligarchic (Rule of the few)
\item The Peloponnesian War 431-404 BCE (Athens surrenders in 404 BCE) This was ultimately the end of the Classical Age
\item Athenian imperial naval power 
\item The Corinthian War 395 BC
\begin{itemize}
\item Sparta + Thebes vs. Athens
\item Athens + Thebes vs. Sparta
\item Sparta + Athens vs. Thebes
\end{itemize}
\item Macedon was a Semi Greek Kingdom
\begin{itemize}
\item Traditional Monarchy
\item Powerful local elites
\item Border instability
\item Coups and assassinations in early 4th Century
\end{itemize}
\item The Battle of Chaironeia 338 BC, Thebean's got outsmarted by Alexander son of Philip
\item Battle of Issos 333 BCE
\item Battle of Gaugamela 331 BCE
\item Seleukid Repression -> Jewish rebellion -> Independent kingdom of Judaea
\item Ptolemaic Egypt: Defensive imperialist at first
\item Friendship with Rome
\item Promote Hellenic culture, literature, arts
\end{itemize}




\textbf{Important People and Accomplishments}
\begin{itemize}
\item Heinrich Schliemann, 1822-1890
\begin{itemize}
\item Archaeologist who was a pioneer in the study of Aegean civ and the Bronze Age
\item The Trojan War
\item Troy
\item Mycenae
\end{itemize}
\item Sir Arthur Evans 1851-1941
\begin{itemize}
\item Another archaeologist and friend of Schliemann
\item Knossos at the beginning of 1900s and other sites on Crete
\end{itemize}
\item Lycurgus, the mythical Spartan lawgiver
\item Philip II of Macedon 359-336 BCE
\begin{itemize}
\item Expanded Macedon's borders
\item Central monarchy
\item Military reforms
\item Marriage alliance with Olympias of Epiros 357 BC
\end{itemize}
\item Darius III 336-330 BCE Emperor of Persia
\item Philip II Assassinated in 336 BC
\item Alexander the Third accession to King
\item 334 BCE was the start of Persian campaign
\item Babylon: "Pharaoh of Egypt, Great King of Persia, King of Sumer, and Akkad, Lord of the Universe"
\item 10 June 323 BC Death of Alexander the Great: his heirs were his son Alex and brother Arrhidaios both were unfit to rule
\item Alexander's generals took over, Antigonos, Seleukos, Ptolemy, The Greek Age: 323-30 BCE
\item Antiochs IV and the Jews 160s BC
\item Ptolemaic Egypt 322-30 BCE
\item Ptolemy I 322-283 BCE
\item Cleopatra VII 51-30 BCE
\end{itemize}
\end{document}