\documentclass[12pt,a4paper]{report}
\usepackage{graphicx}
\usepackage{amsmath}
\usepackage{amsfonts}
\usepackage{amssymb, mathpazo}

\begin{document}

	\centering
	{\scshape\LARGE CLAS100 \par}
	{\scshape\Large Lecture 7\par}
	%{\Large\itshape Yaro Gorban\par}
	{\large \today\par}
	\vspace{1.5cm}
	
\textbf{Preview: Roman Constitutions}
\begin{itemize}
\item The Monarchy: 753- c. 500 BC
\begin{itemize}
\item The expulsion of Etruscan royal dynasty
\end{itemize}
\item The Republic: c.500 - 30 BC
\begin{itemize}
\item there was a peoples assemblies and senates
\item more elitist than the greeks
\end{itemize}
\item The Empire: 30 BC - 476 AD
\end{itemize}
\textbf{Republican Society as the Agent of Its Own Destruction:}
\begin{itemize}
\item The Romans hated the idea of having a king
\item In Roman Society competition for status, power and glory was considered a positive trait
\item Patrons vs. Clients
\begin{itemize}
\item Patron is a wealthy older man with status
\item Young up and comer would be considered a client
\end{itemize}
\item In 146 BC Greece becomes a Roman Colony
\end{itemize}
\textbf{The Dual meaning of the term "Roman Empire"}
\begin{itemize}
\item Empire could refer to all of Rome's holdings
\item or it could refer to a \textbf{constitutional} term, a period of time where Rome was ruled by emperors
\item Augustus was the first emperor
\item The Roman Empire was vast constitutionally
\end{itemize}
\textbf{Citizen Inequities}
\begin{itemize}
\item The rich got richer, the poor got poorer
\item There was an agricultural crisis
\begin{itemize}
\item The rich controlled the supply of grain as well as the price
\end{itemize}
\item There was an employment crisis
\begin{itemize}
\item The aristocrats(the wealthy) hired people, and expected the employees to be indebted to them
\end{itemize}
\item There was a military crisis
\begin{itemize}
\item The army needed to be large to control the vast area of Rome
\item Most soldiers were paid by the people who enlisted them
\end{itemize}
\end{itemize}
\textbf{Non-citizens:}
\begin{itemize}
\item Disaffection of Italian allies
\item 50 percent of the army were forcefully enlisted
\item Slave revolts
\begin{itemize}
\item Spartacus 73-71 BC lead a revolt
\item It went terribly wrong
\end{itemize}
\end{itemize}
\textbf{The Gracchus Brothers}
\begin{itemize}
\item Tiberius, tribune(a representative of the people) 133 BC
\item Tiberius proposed land that was state-owned land
\item Gaius, tribune 123, 122- BC
\item Gaius proposed to regulate grain prices; all allied people should become citizens of Rome
\end{itemize}
\textbf{Rise of the Warlords}
\begin{itemize}
\item Gaius Marius (157-86 BC)
\item Was known for his military doings outside of Rome
\item He hired people who didn't own land to the army (proletarii)
\item Changed the laws that anyone can join the army
\end{itemize}
\textbf{L. Cornelius Sulla (138-78 BC:}
\begin{itemize}
\item Rivalry with Marius
\item Sulla's return to Italy and march on Rome
\item General's fighting outside of Rome, were not allowed to enter Rome
\item Doing so meant a declaration of war
\end{itemize}
\textbf{Rise of Julius Caesar (100-44 BC):}
\begin{itemize}
\item There was a rule of a trio
\item M. Licinius
\item J. Caesar
\item Pompey "The Great"
\item Caesar Succeeds in Gaul
\item Caesar's daughter Julia Died, Pompey's wife
\item Crassus died in Parthia
\item Civil war in 49-45 BC
\end{itemize}
\end{document}