\documentclass[12pt,a4paper]{report}
\usepackage{graphicx}
\usepackage{amsmath}
\usepackage{amsfonts}
\usepackage{amssymb, mathpazo}

\begin{document}

	\centering
	{\scshape\LARGE CLAS100\par}
	{\scshape\Large Lecture 14\par}
	%{\Large\itshape Yaro Gorban\par}
	{\large October 19, 2015}
	\vspace{1.5cm}
	
\textbf{Greek Theatre, Religion and Tragedy}
\begin{itemize}
\item The origin of Greek Tragedy
\item Aristotle's Poetics:
\item Choral hymns in honour of Dionysus
\item Myths of the gods; hymns, songs and dance
\item Hypokrites: interpreter
\item Tragoidoi: goat-singers
\end{itemize}

\textbf{Dramatic Festivals}
\begin{itemize}
\item Lenaea(Jan-Feb) in honour of Dionysos Lenaios
\item Anthesteria(Feb-Mar) in honour of Dionysus
\item The Dionysia was a large festival in honour of Dionysus
\end{itemize}

\textbf{The Theatre of Athens}
\begin{itemize}
\item Established in 6th century BC
\item Seated 14000-17000 people
\item 499 BC: burned to the ground
\item Dionysus Eletherius theatre
\item At the center of the row surrounding the orchestra there was a seat reserved for the priest or Dionysus
\end{itemize}

\textbf{Theatre of Epidauros}
\begin{itemize}
\item Oldest and most preserved ancient theatre
\item came about around 340 BC
\item 13000 capacity
\end{itemize}

\textbf{Sophocles}
\begin{itemize}
\item 495-406 BC
\item Classical playwright: justice, democracy, civilization, control, beauty
\item Hubris: excessive pride or self-confidence
\item Peripeteia: change of circumstances
\item Panhellenism following Persian wars
\item Nemesis: the spirit of divine retribution against those who succumb to hubris
\item Oedipus Tyrannus A tragedy written by Sophocles
\end{itemize}

\textbf{Euripides}
\begin{itemize}
\item 480-406 BC
\item Playwright of change and decline
\item Peloponnesian War
\item Challenge accepted religious, social, political values
\item Mind vs. heart; control vs. freedom; tradition vs. change
\item Bacchae; A tragedy set in Thebes
\end{itemize}
\end{document}
