\documentclass[12pt,a4paper]{report}
\usepackage{graphicx}
\usepackage{amsmath}
\usepackage{amsfonts}
\usepackage{amssymb, mathpazo}

\begin{document}

	\centering
	{\scshape\LARGE CLAS100 \par}
	{\scshape\Large Lecture 7\par}
	%{\Large\itshape Yaro Gorban\par}
	{\large \today\par}
	\vspace{1.5cm}
	
\textbf{The Etruscans: Reputation and Reality}
\begin{itemize}
\item Ancient Etruria, occupied Italy
\item Several tribes occupied Italy before the Romans consolidated them
\item Religion took a lot from Greek culture
\end{itemize}
\textbf{Development of Etruscan Civilization:}
\begin{itemize}
\item Highpoint in the 7$^{th}$/6$^{th}$ centuries BC
\item traded with Greeks and Phoenicians
\item Controlled Rome until 508 BC
\item After 500 BC BC rollback in Etruscan power
\begin{itemize}
\item Romans destroyed an important city Veii 396 BC
\item Rome absorbed Etruscan cities by second century
\end{itemize}
\end{itemize}
\textbf{Etruscan through a Greco-Roman Lens:}
\begin{itemize}
\item Tyrrhenoi(Greek)/Etrusci(Roman) - Rasna(by themselves)
\item Source Problems:
\begin{itemize}
\item When the Greeks referred to the Tyrrhenoi, noone was sure if they talked about the Etruscans or people who lived in that area
\end{itemize}
\end{itemize}
\textbf{Theopompos on Etruscan Customs:}
\begin{itemize}
\item They lived a very lavish lifestyle
\item They shared their women and wives
\item Low standard of morality, especially among women
\end{itemize}
\textbf{Wealthy Etruscan Society:}
\begin{itemize}
\item The decadent world portrayed by Theopompos?
\item or a society that enjoyed the good things in life
\end{itemize}
\textbf{THe Etruscan Legacy:}
\begin{itemize}
\item Political symbols, such as fasces(twine with a single headed axe)
\item Religion and Alphabet passed down to the Romans
\end{itemize}
\end{document}